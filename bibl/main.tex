\documentclass[11pt]{sdm}

\usepackage{graphicx}
\graphicspath{{../img/}}

\pagestyle{plain}

\usepackage{hyperref}
\usepackage{url} \urlstyle{sf}
\newcommand{\email}[1]{\href{mailto:#1}{#1}}

\usepackage[dvipsnames]{xcolor}
\newcommand{\addref}[1]{\colorbox{TealBlue!100}{\textcolor{white}{\textbf{$[$\ifx&#1&\ \else#1\fi$]$}}}}
\newcommand{\todo}[1]{\colorbox{Red!75}{\textcolor{white}{\textbf{TODO\ifx&#1&\else: #1\fi}}}}
\newcommand{\done}{\colorbox{YellowGreen!100}{\textcolor{white}{\textbf{DONE}}}}
\newcommand{\review}{\colorbox{YellowOrange!100}{\textcolor{white}{\textbf{REVIEW}}}}

\title{Search-Based Test Amplification}

\author{Simon \textsc{Bihel}}
\supervisorOne{Benoit \textsc{Baudry}}
\supervisorTwo{}
\team{KTH Royal Institute of Technology}
\school{ens-Rennes}

\domain{Domain: Software Engineering}

\abstract{write your abstract here}

\date{February 2018}

\begin{document}
\maketitle

\section*{Introduction}
\label{intro}

With the poor use of formal specifications for software development, programmers ensure the well-behavior of software with hand-written tests.
Meta-heuristic optimizing techniques are used to automate the process of testing (e.g.\ generating test data).
% target function is a software metric

With practices such as test-driven development, software projects come with strong test suites.
Knowledge of the expected properties of the program can be extracted from the large number of test cases.
This knowledge can be used in turn to enhance the test suite to improve certain software metrics.

Different goals can be pursued to reduce the likelihood of bugs, to help during debugging, etc.
The amplification can be achieved by generating variants of existing tests, modifying the tests execution, etc.
\todo{badly written}

After explaining the basics of Search-Based Software Testing in Section~\ref{sbse} we will present how it can be used to amplify test suites in Section~\ref{tsa}.
In Section~\ref{planned} we detail the different paths the internship could explore.


\section{Search-Based Software Engineering}
\label{sbse}
\todo{intro}

\subsection{Motivations}
\label{motiv}
\todo{}

\cite{mcminn2011search}

\subsection{Search-Based Optimization Algorithms}
\label{example_algo}
\todo{}

\subsection{Application Areas}
\label{applications}
\todo{}

\cite{danglot2017emerging}
\cite{xuan2015dynamic}


\section{Test Suite Amplification}
\label{tsa}
\todo{intro}

\addref{new survey paper}

\subsection{Sosies Synthesis}
\label{sosies}
\todo{}

\cite{baudry2015dspot,baudry2015automatic,baudry2014tailored}

\subsection{Test Suite Evaluation}
\label{testsuite_eval}
\todo{}

\addref{new dspot paper}

\subsection{Related Works}
\label{related}
\todo{}

\cite{yoo2012test}
\cite{fraser2011evosuite}


\section{Planned Work}
\label{planned}
\todo{intro}

\subsection{Evaluation of the Added Value of Hand-Written Tests}
\label{evaluation}
\todo{}

\subsection{Using Literals Found in Source Code}
\label{mutation}
\todo{}

\subsection{Adding Explanations}
\label{explanation}
\todo{}


\section*{Conclusion}
\label{conclu}
\todo{recall what will be done, in what context}


\bibliographystyle{ieeetr}
\bibliography{bibl}

\end{document}
