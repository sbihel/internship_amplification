\documentclass[11pt]{sdm}

\usepackage{graphicx}
\graphicspath{{../img/}}

\pagestyle{plain}

\usepackage{hyperref}
\usepackage{url} \urlstyle{sf}
\newcommand{\email}[1]{\href{mailto:#1}{#1}}

\usepackage[dvipsnames]{xcolor}
\newcommand{\addref}[1]{\colorbox{TealBlue!100}{\textcolor{white}{\textbf{$[$\ifx&#1&\ \else#1\fi$]$}}}}
\newcommand{\todo}[1]{\colorbox{Red!75}{\textcolor{white}{\textbf{TODO\ifx&#1&\else: #1\fi}}}}
\newcommand{\done}{\colorbox{YellowGreen!100}{\textcolor{white}{\textbf{DONE}}}}
\newcommand{\review}{\colorbox{YellowOrange!100}{\textcolor{white}{\textbf{REVIEW}}}}

\title{Search-Based Test Amplification}

\author{Simon \textsc{Bihel}}
\supervisorOne{Benoit \textsc{Baudry}}
\supervisorTwo{}
\team{KTH Royal Institute of Technology}
\school{ens-Rennes}

\domain{Domain: Software Engineering - Artificial Intelligence}

\abstract{%
With the poor use of formal specifications for software development, programmers ensure the well-behavior of software with hand-written tests.%
Meta-heuristic optimizing techniques are used to automate the process of testing (e.g.\ generating test data).%
%
With practices such as test-driven development, software projects come with strong test suites.%
Knowledge of the expected properties of the program can be extracted from the large number of test cases.%
This knowledge can be used in turn to enhance the test suite to improve certain software metrics.%
%
Different goals can be pursued to reduce the likelihood of bugs, to help during debugging, etc.%
The amplification can be achieved by generating variants of existing tests, modifying the tests execution, etc.%
}

\date{February 2018}

% Ought to be 10 to 15 pages long

\begin{document}
\maketitle

\section*{Introduction}
\label{intro}

With the poor use of formal specifications for software development, programmers ensure the well-behavior of software with hand-written tests.
Meta-heuristic optimizing techniques are used to automate the process of testing (e.g.\ generating test data).
% target function is a software metric

With practices such as test-driven development, software projects come with strong test suites.
Knowledge of the expected properties of the program can be extracted from the large number of test cases.
This knowledge can be used in turn to enhance the test suite to improve certain software metrics.

Different goals can be pursued to reduce the likelihood of bugs, to help during debugging, etc.
The amplification can be achieved by generating variants of existing tests, modifying the tests execution, etc.
\todo{badly written}

After explaining the basics of Search-Based Software Testing in Section~\ref{sbse} we will present how it can be used to amplify test suites in Section~\ref{tsa}.
In Section~\ref{planned} we detail the different paths the internship could explore.


\section{Search-Based Software Engineering}
\label{sbse}

In this section we present the reasons for the emergence of Search-Based Software Engineering (SBSE) and give examples of applications.

\subsection{Motivations}
\label{motiv}
\todo{}

Software projects are continuously getting more complex.
They often rely on informal specifications that change during the evolution of the project.
Because of the scale and weak structure, formal methods cannot be used in many cases.

In order to tackle problems such as optimization or testing, we need insightful approximating methods.

Over time software engineers have developed many metrics to evaluate a piece of software and get feedback on its quality.

\cite{mcminn2011search}

\subsection{Search-Based Optimization Algorithms}
\label{example_algo}

To reach a software engineering goal for these complex systems we use \textit{soft-computing} (or \textit{computational intelligence}) to find inexact or sub-optimal solutions in a reasonable time.
In~\ref{basic_algo} we present some algorithms to give an idea of the capabilities of these techniques.
In~\ref{fitness_func} we show how we can pursue engineering goals with these optimization processes.

\subsubsection{Basic algorithms}
\label{basic_algo}
\todo{}

\subsubsection{Metrics as fitness functions}
\label{fitness_func}
\todo{}

\begin{itemize}
  \item basic metrics such as code coverage
  \item mutation score
\end{itemize}

\subsection{Application Areas}
\label{applications}
\todo{Remove this part? Superfluous}

\begin{itemize}
  \item test data generation
  \item test data regeneration
\end{itemize}

\cite{danglot2017emerging}
\cite{xuan2015dynamic}


\section{Test Suite Amplification}
\label{tsa}
In this section we present the specific problem of enhancing an existing test suite.
First,~\ref{motiv_tsa} exposes the motivations for trying to amplify a preexisting and seemingly strong test suite.
Then~\tef{related} presents a few papers to give an idea of the related works.
The works that the internship will be based on are presented in~\ref{sosies} and~\ref{testsuite_eval}.

\addref{new survey paper}

\subsection{Motivations}
\label{motiv_tsa}
\todo{}

Important projects from big companies now come with extensive test suites thanks to good practices.
It is thus reasonable to try and see if we can push even further the quality of such projects.

We can see the test suite as a starting population of already good quality for our evolutionary algorithms.
But most importantly, we can use the test suite as a set of specifications that give us knowledge about what the software is supposed to do and thus allow us to detect more bugs.
It gives us an \textit{oracle}.

Another thing we can do is enhance the test suite to make it better for certain metrics.

\subsection{Related Works}
\label{related}
\todo{}

\subsubsection{Test Data Regeneration}
\cite{yoo2012test}
\todo{}

\subsubsection{B-Refactoring}
\cite{xuan2016b}
\todo{}

\subsubsection{Evosuite}
\cite{fraser2011evosuite}
\todo{}


\subsection{Sosies Synthesis}
\label{sosies}
\todo{}

\begin{itemize}
  \item same outputs for specified input domain, different output outside
\end{itemize}

\cite{baudry2015dspot,baudry2015automatic,baudry2014tailored}

\subsection{Test Suite Amplification}
\label{testsuite_eval}
\todo{}

\begin{itemize}
  \item
\end{itemize}

\addref{new dspot paper}


\section{Planned Work}
\label{planned}
This section explains the different paths that we could explore during the internship.

\subsection{Evaluation of the Added Value of Hand-Written Tests}
\label{evaluation}
\todo{}

\begin{itemize}
  \item explain that no objective evaluation has been made to prove the value of having a pre-existing test suite
  \item explain what kind of experiment
\end{itemize}

\subsection{Using Literals Found in Source Code}
\label{mutation}
\todo{}

\begin{itemize}
  \item Explanation of mutation operators for strings
  \item how we could collect literals
  \item what kind of experiment
\end{itemize}

\subsection{Creating mutation operators for specific data structures}
\label{create_operators}
\todo{}

\begin{itemize}
  \item
\end{itemize}

\subsection{Stacking mutations}
\label{stacking}
\todo{}

\begin{itemize}
  \item
\end{itemize}

\subsection{Adding Explanations}
\label{explanation}
\todo{}

\begin{itemize}
  \item explain the problem of understanding PRs
\end{itemize}

\subsection{Learning the set of good amplification operators}
\label{learning}
\todo{}

\begin{itemize}
  \item
\end{itemize}

\subsection{Reduce the amplified tests to a minimal set of useful tests}
\label{minimal}
\todo{}

\begin{itemize}
  \item related to `add new test method or replace another one'
\end{itemize}


\section*{Conclusion}
\label{conclu}
\todo{recall what will be done, in what context}


\bibliographystyle{ieeetr}
\bibliography{bibl}

\end{document}
